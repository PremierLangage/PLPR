\documentclass[border=10pt]{article}

\usepackage[francais]{babel}
\usepackage[utf8]{inputenc}
\usepackage[T1]{fontenc}

% espace entre les lignes.
\linespread{1.1}

% espacement entre les paragraphes.
\setlength{\parindent}{10pt}

\usepackage{amsmath}
\usepackage{amssymb}
\usepackage[mathcal]{eucal}
\usepackage[dvips]{graphics}

%\usepackage{theorem}

\usepackage{color}

\usepackage{listings}
\usepackage{pxfonts}

% Paramètres de listings.
\lstset{
    language=C,
    numbers=left,
    numberstyle=\footnotesize,
    stepnumber=1
}

\usepackage{tikz}
\usetikzlibrary{shapes,shadows,arrows,positioning,graphs}
\usetikzlibrary{calc,decorations.pathmorphing,intersections}
\usetikzlibrary{arrows.meta}
\tikzset{%
  >={Latex[width=2mm,length=2mm]},
  % Specifications for style of nodes:
            base/.style = {rectangle, rounded corners, draw=black,
                           minimum width=4cm, minimum height=1cm,
                           text centered, font=\sffamily},
          acteur/.style = {base, fill=yellow!30},
            data/.style = {base, fill=blue!30},
         display/.style = {base, fill=green!30},
         process/.style = {base, fill=red!30},
}


\title{Structuration dans Premier Langage}
\author{DR et all.}

\begin{document}    


% Drawing part, node distance is 1.5 cm and every node
% is prefilled with white background
\hspace{-2.5cm}\begin{tikzpicture}[node distance=1.5cm,
    every node/.style={fill=white, font=\sffamily}, align=left]

  \node (exo) [data] {\underline{Exercice \textbf{.pl}} \\ fichier déclaratif \\ * Titre \\ * Énoncé \\ * Formulaire \\ * ...};
  \node (play) [data, below of=exo, yshift=-3cm] {\underline{Exercice préparé} \\ même fichier \\ \\ champs en plus \\ valeurs modifiées \\ };
  \node (eval) [data, below of=play, yshift=-3cm] {\underline{Exercice corrigé} \\ même fichier \\ \\ note générée \\ feedback \\ validation};

  \draw (exo) edge[->, bend left=80] (play);
  \draw (play) edge[->, bend left=80] (eval);

  \node (com) [right of=exo, xshift=4.5cm, yshift=-2.25cm] {pré-traitement \\ tirage des parties aléatoires \\ Builder};
  \node (com) [right of=play, xshift=4cm, yshift=-2.25cm] {post-traitement \\ notation et correction \\ Grader};

  \node (brut) [left of=exo, xshift=-5cm] {Produit par \\ un enseignant \\ en amont};
  \draw (brut) edge[->] (exo);

  \node (aff) [left of=play, xshift=-5cm] {Affichage de l'exercice \\ en attente \\ des réponses élèves};
  \draw (play) edge[->] (aff);

  \node (feed) [left of=eval, xshift=-5cm] {Un feedback personnalisé \\ est donné \\ à l'élève};
  \draw (eval) edge[->] (feed);


\end{tikzpicture}

\end{document}
